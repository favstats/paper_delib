\documentclass[]{article}
\usepackage{lmodern}
\usepackage{amssymb,amsmath}
\usepackage{ifxetex,ifluatex}
\usepackage{fixltx2e} % provides \textsubscript
\ifnum 0\ifxetex 1\fi\ifluatex 1\fi=0 % if pdftex
  \usepackage[T1]{fontenc}
  \usepackage[utf8]{inputenc}
\else % if luatex or xelatex
  \ifxetex
    \usepackage{mathspec}
  \else
    \usepackage{fontspec}
  \fi
  \defaultfontfeatures{Ligatures=TeX,Scale=MatchLowercase}
\fi
% use upquote if available, for straight quotes in verbatim environments
\IfFileExists{upquote.sty}{\usepackage{upquote}}{}
% use microtype if available
\IfFileExists{microtype.sty}{%
\usepackage{microtype}
\UseMicrotypeSet[protrusion]{basicmath} % disable protrusion for tt fonts
}{}
\usepackage[margin=1in]{geometry}
\usepackage{hyperref}
\hypersetup{unicode=true,
            pdfborder={0 0 0},
            breaklinks=true}
\urlstyle{same}  % don't use monospace font for urls
\usepackage{graphicx,grffile}
\makeatletter
\def\maxwidth{\ifdim\Gin@nat@width>\linewidth\linewidth\else\Gin@nat@width\fi}
\def\maxheight{\ifdim\Gin@nat@height>\textheight\textheight\else\Gin@nat@height\fi}
\makeatother
% Scale images if necessary, so that they will not overflow the page
% margins by default, and it is still possible to overwrite the defaults
% using explicit options in \includegraphics[width, height, ...]{}
\setkeys{Gin}{width=\maxwidth,height=\maxheight,keepaspectratio}
\IfFileExists{parskip.sty}{%
\usepackage{parskip}
}{% else
\setlength{\parindent}{0pt}
\setlength{\parskip}{6pt plus 2pt minus 1pt}
}
\setlength{\emergencystretch}{3em}  % prevent overfull lines
\providecommand{\tightlist}{%
  \setlength{\itemsep}{0pt}\setlength{\parskip}{0pt}}
\setcounter{secnumdepth}{0}
% Redefines (sub)paragraphs to behave more like sections
\ifx\paragraph\undefined\else
\let\oldparagraph\paragraph
\renewcommand{\paragraph}[1]{\oldparagraph{#1}\mbox{}}
\fi
\ifx\subparagraph\undefined\else
\let\oldsubparagraph\subparagraph
\renewcommand{\subparagraph}[1]{\oldsubparagraph{#1}\mbox{}}
\fi

%%% Use protect on footnotes to avoid problems with footnotes in titles
\let\rmarkdownfootnote\footnote%
\def\footnote{\protect\rmarkdownfootnote}

%%% Change title format to be more compact
\usepackage{titling}

% Create subtitle command for use in maketitle
\newcommand{\subtitle}[1]{
  \posttitle{
    \begin{center}\large#1\end{center}
    }
}

\setlength{\droptitle}{-2em}
  \title{}
  \pretitle{\vspace{\droptitle}}
  \posttitle{}
  \author{}
  \preauthor{}\postauthor{}
  \date{}
  \predate{}\postdate{}


\begin{document}

Since political theory took its ``deliberative turn''
{[}@dryzek2000deliberative{]} in the 1990s, political science has
increasingly turned towards empirically examining deliberation. There
have been numerous studies about its requirements and consequences. This
paper is concerned with the latter. Deliberative theory along with
empirical science has developed manifold assumptions about the effects
of deliberation, including transformation of preferences, epistemic
quality, consensus and accommodation, as well as side-effects on civic
virtues like political trust {[}cf. @bachtiger2013empirische pp.
164-165{]}. Given the current decline of confidence in governments and
political institutions in many democracies across the world {[}cf.
@foa2016democratic{]}, deliberation could be seen as a process to arrive
at legitimate decisions in societies of increasing complexity {[}see for
example @habermas1994three pp. 7-8; @warren2015can p. 562{]}. In order
to examine this legitimacy claim, this paper seeks to investigate
whether deliberation increases citizens perception of regime legitimacy,
which is conceptualized and operationalized as regime support. This
study differs from previous ones in the following terms: it is the first
to examine the effects of deliberation on regime support in a
cross-national framework across a large dataset of 316,938 respondents
from 119 countries across all continents. Moreover, the analysis is not
restricted to democratic regimes, but also includes
non-democracies.\footnote{For the purposes of this paper we consequently
  refer to political systems as non-democratic in accordance with the
  Polity IV project classification of autocracies and anocracies.} In
order to account for the variety of regime types, we draw from the
literature on so called \textit{authoritarian deliberation}, a recent
theoretical development that conceptualizes deliberation outside of
democratic contexts {[}see @he2014deliberative; @he2011authoritarian;
@he2010giving{]}.

The main research question of this thesis states as follows:
\textit{What role does Deliberation play for regime legitimacy across the world?}
The following section reviews relevant literature on deliberation and
derives assumptions to be tested empirically (Section \ref{theory}). The
next section presents the research design of this study and discusses
issues critical to the analysis - especially the validity of the
Deliberative Component Index as well as a possible bias in self-reported
regime support. Following this, bivariate relationships are examined and
the results of the estimated multilevel models are presented and
interpreted in regard to their implications for the theoretical
assumptions (Section \ref{empirical}). In the end, the findings of the
analysis will be summarized and the conclusion gives an answer to the
research question along with a discussion of implications for further
research (Section \ref{conclusion}).


\end{document}
